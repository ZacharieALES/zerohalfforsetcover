\documentclass[12pt]{report}

\usepackage[francais]{babel}
\usepackage[utf8]{inputenc}
\usepackage[T1]{fontenc}
\usepackage{amsmath}

\usepackage[cyr]{aeguill}
\usepackage{fancyheadings}
\usepackage[pdftex]{graphicx}
\DeclareGraphicsExtensions{.jpg,.pdf,.png}
\usepackage[pdftex,colorlinks=true,linkcolor=blue,citecolor=blue,urlcolor=blue]{hyperref}
\usepackage{anysize}
\marginsize{22mm}{14mm}{12mm}{25mm}
\usepackage{natbib}
\usepackage{icomma}


\begin{document}
\pagestyle{fancyplain}
\renewcommand{\chaptermark}[1]{\markboth{\chaptername\ \thechapter. #1}{}}
\renewcommand{\sectionmark}[1]{\markright{\thesection. #1}}
\lhead[]{\fancyplain{}{\bfseries\leftmark}}
\rhead[]{\fancyplain{}{\bfseries\thepage}}
\cfoot{}

%% Voilà mes légendes de figures comme je les aime :
\makeatletter
\def\figurename{{\protect\sc \protect\small\bfseries Fig.}}
\def\f@ffrench{\protect\figurename\space{\protect\small\bf \thefigure}\space}
\let\fnum@figure\f@ffrench%
\let\captionORI\caption
\def\caption#1{\captionORI{\rm\small #1}}
\makeatother

\graphicspath{{img/}}

%%%%%%%%%%%%%%%%%%%%%%%%%%%%%%%%%%%%%%%%%%%%%%%%%%%%%%%%%% Couverture :
\thispagestyle{empty}
{\Large
\begin{center}
Prénom NOM
\vskip1cm

%% Pour redéfinir la distance entre la boite et le texte
\fboxsep6mm
%% Pour redéfinir l'épaisseur de la boite
\fboxrule1.3pt

%% Le \vphantom{\int_\int} sert à introduire de l'espace entre les deux lignes
%% (essayez donc de le commenter)
$$\fbox{$
  \begin{array}{c}
  \textbf{Titre}
  \vphantom{\int_\int}
  \end{array}
  $}
$$
\end{center}
\vskip8cm

\begin{flushright}
\textit{Encadrant :}

Zacharie ALES
\end{flushright}
}

\clearpage

%%%%%%%%%%%%%%%%%%%%%%%%%%%%%%%%%%%%%%%%%%%%%%%%%%%%%%%%%% Table des matières :
\renewcommand{\baselinestretch}{1.30}\small \normalsize

\tableofcontents

\renewcommand{\baselinestretch}{1.18}\small \normalsize


%%%%%%%%%%%%%%%%%%%%%%%%%%%%%%%%%%%%%%%%%%%%%%%%%%%%%%%%%% Introduction :
\chapter{Introduction}

\section{Exemples}

\subsection{Bases}

Text normal.

\textit{Texte italique.}

\textbf{Texte en gras.}

\underline{Texte souligné.}

\subsection{Listes}

Liste sans numéro :
\begin{itemize}
\item item 1 ;
\item item 2.
\end{itemize}

Liste avec numéro :
\begin{enumerate}
\item item 1 ;
\item item 2.
\end{enumerate}

\subsection{Formules mathématiques}

Exemple de formule mathématique :
$ x_j + \alpha_j = \varepsilon_1 + z^2 + \frac 1 2 + \sum_{i=1}^n w_i\quad \forall j\in\{1,
..., p\}$

Exemple d'équation numérotée :

\begin{equation}
x = y
  \label{eq:monEquation}
\end{equation}

Référence à cette équation : \eqref{eq:monEquation}.

\subsection{Tableaux}

La table~\ref{tab:ex} contient trois colonnes :
\begin{itemize}
\item la première est centrée à gauche('l' : left) ;
\item la seconde est centrée ('c' : center) ;
\item la troisième est centrée à droite ('r' : right)
\end{itemize}
\begin{table}[h!]
  \centering
\begin{tabular}{lcr}
\hline % Représente une ligne horizontale (doit toujours être en début
       % de ligne

\textbf{Titre 1} & \textbf{Titre 2} & \textbf{Titre 3} \\
\hline

c1 & c2 & c3 \\
c4 & c5 & c6\\
\hline
\end{tabular}

\caption{Exemple de table.}
\label{tab:ex}
\end{table}


\subsection{Figures}

La figure~\ref{fig:maFigure} représente le logo de l'ENSTA.

\begin{figure}[h!]
  \centering
  \includegraphics[height=4cm]{Logo_ENSTA_Paris.png}
  \caption{Légende}
  \label{fig:maFigure}
\end{figure}

\subsection{Références bibliographiques}
 
Pour citer un article en latex il faut :
\begin{enumerate}
\item  trouver  le  fichier  .bib  associé  à  cet  article  (on  peut
  généralement le trouve sur le site google
  scholar) ;
\item   copier   le   contenu   de  ce   fichier   dans   le   fichier
  "bibliography.bib" ;
\item citer ce fichier en  utilisant son identifiant (l'identifiant se
  trouve au début du fichier .bib).
\end{enumerate}

Par  exemple, pour  créer la  référence~\cite{bertsimas2017optimal}, il
faut taper :

$\backslash$cite\{bertsimas2017optimal\}.



\addcontentsline{toc}{chapter}{Bibliographie}

%% Feuille de style bibliographique : monjfm.bst
\bibliographystyle{apalike}
\bibliography{bibliography}

\end{document}
